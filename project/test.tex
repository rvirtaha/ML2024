\documentclass{article}
\author{Risto Virtaharju and Olli Mustajoki}
% Set page size and margins
% Replace `letterpaper' with `a4paper' for UK/EU standard size
\usepackage[a4paper,top=2cm,bottom=2cm,left=3cm,right=3cm,marginparwidth=1.75cm]{geometry}


% Emerald Harvard Citation Style

\usepackage[style=authoryear,backend=biber,natbib=true,maxcitenames=2,uniquelist=false]{biblatex}
\addbibresource{Bibliography.bib} % your .bib file


\begin{document}
%Front Matter

\title{A Machine Learning Approach to Detecting Classifying Billiards Balls from Images}
\maketitle


\begin{abstract}
\textbf{Purpose} - This study aims to investigate the impact of [specific variable or phenomenon] on [specific outcome or field], addressing a critical gap in the existing literature. The research seeks new insights into [specific problem or issue], contributing to a deeper understanding of [relevant context or application].

\textbf{Design/Methodology/Approach} - The study employs aach, utilizing [specific methodologies such as surveys, experiments, case studies, etc.]. Data were collected from [description of the sample or data sources] and analyzed using [specific analytical tools or software].

\textbf{Findings} - The results reveal that [briefly describe key findings], indicating that [specific conclusion or implication]. These findings provide evidence that [mention whether the hypothesis was confirmed or if new patterns emerged], offering significant implications for [specific field, industry, or theoretical framework].

\textbf{Originality/Value} - This research offers a novel perspective on [topic], providing valuable insights that extend the current understanding of [subject matter]. The study’s findings contribute to [mention the advancement or practical application], highlighting areas for future research and potential policy or practical applications. 

\end{abstract}

\section*{Abstract}
Harvard citation style defined by Emerald Publishing, along with section styles and other adjustments required for submission in Construction Innovation journal.
Users should adapt the styling to align with the guidelines of the journal to which they are submitting their article. Make sure to remove the authors' names and acknowledgements if you are submitting an anonymous file for double-blind peer review.

\section{Introduction}
\label{sec:introduction}


background and context for your study, highlighting the importance and relevance of the topic. 
Then, clearly identify the research problem or gap in your study's existing literature, explaining why it is significant. 
State the research questions or objectives your study aims to answer, ensuring they are directly linked to the identified problem. 
Justify the need for your research by discussing its potential contributions or impact on the field.
You may also include a brief overview of the methodology, especially if it's novel or crucial to your study's contribution. 
Additionally, define the scope and limitations of your research, clarifying what the study will and will not cover.
If applicable, present your main thesis statement or hypothesis. 
Optionally, you can conclude the introduction with a brief outline of the paper's structure to guide the reader.

\section{Method}
\label{sec:method}and detailed description of the research procedures employed in the study. 
It should explain the data collection and analysis techniques, ensuring that the description is thorough enough to allow others to replicate the study. 
The section should be written in the past tense and typically employ passive voice to maintain an objective tone. 
Additionally, it is essential to include all necessary bibliographic information for each source cited, such as in the example \cite{fellows_research_2021}.

Use the Emerald Harvard style defined in this template for references, and adjust the citation style if necessary. Include DOIs where available. 

\section{Main Section}
\label{sec:main_section}
When submitting manuscripts using \LaTeX, a PDF file must also be included. 
Articles should be between 6,000 and 10,000 words, encompassing all text, tables, and figures. 
A concise and descriptive title must be included, and it is essential to list all contributing authors in the submission, along with their email addresses, names, and affiliations.




\subsection{Subsection 2}

\textbf{Emerald Publishing} accepts formats such as .ai, .eps, .jpeg, .bmp, and .tif. Electronic figures created in other applications should be provided in their original formats. 
Additionally, these figures should either be copied and pasted into a blank MS Word document or submitted as a PDF file.

\section{Results}
\label{sec:modelling}
This \LaTeX template is designed to create an academic article with specific formatting and citation requirements. Here's a plain-text explanation of how the template has been structured:

Useful Packages:
        Several \LaTeX packages are included to provide additional functionalities:
            \begin{itemize}
                \item amssymb: Provides access to additional mathematical symbols.
                \item siunitx: Allows for consistent formatting of numbers and units.
                \item hyperref: Adds hyperlinks to the document, with an option to handle URL line breaks.
                \item cleveref: Enhances cross-referencing features.
                \item inputenc: Handles input encoding, using UTF-8 for wider character support.
                \item lineno: Adds line numbers to the document.
                \item csquotes: Handles quotation marks in a language-sensitive way.
                \item booktabs: Provides better formatting for tables.
                \item longtable: Supports tables that can span multiple pages.
                \item adjustbox: Allows for resizing and adjusting the positioning of content.
                \item array: Provides additional tools for formatting tables and arrays.
                \item url: Simplifies the inclusion of URLs in the document.
                \item titlesec: Customizes section titles (though the specific customization is commented out).
                \item authblk: Manages author affiliations.
            \end{itemize}


\section{Discussion}
\label{sec:discussion}encompasses several elements. 
First, it involves the interpretation of results, where you explain what your findings signify in relation to your research question. 
This is followed by a comparison with previous work, which involves relating your results to existing literature and highlighting similarities and differences. 
If applicable, you should also explain any unexpected results, discussing surprising findings and their potential causes. 
Additionally, this section should address the broader implications of your results for the field or related areas. 
Acknowledging any limitations of your study or methodology is crucial, as is suggesting potential avenues for future research based on your findings.

\section{Conclusion}
\label{sec:conclusion}
reinforcing the core message of the paper. This is followed by a summary of the key points discussed throughout the text, which helps to consolidate the main arguments. The conclusion should also include final thoughts or reflections on the topic, providing a thoughtful discussion wrap-up. 

\hfill

\textit{Acknowledgments} \\Mention all external funding sources in the acknowledgements.

\textit{Disclosure statement:} \\No potential conflict of interest is reported by the authors




%\break
\printbibliography

\end{document}